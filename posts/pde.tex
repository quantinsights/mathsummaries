%% LyX 2.3.0 created this file.  For more info, see http://www.lyx.org/.
%% Do not edit unless you really know what you are doing.
\documentclass[english]{article}
\usepackage{amsmath}
\usepackage{amsthm}
\usepackage{fontspec}
\setlength{\parskip}{\medskipamount}
\setlength{\parindent}{0pt}
\usepackage{color}
\usepackage[unicode=true,pdfusetitle,
 bookmarks=true,bookmarksnumbered=true,bookmarksopen=false,
 breaklinks=false,pdfborder={0 0 1},backref=false,colorlinks=true]
 {hyperref}

\makeatletter
%%%%%%%%%%%%%%%%%%%%%%%%%%%%%% Textclass specific LaTeX commands.
\numberwithin{equation}{section}
\theoremstyle{definition}
\newtheorem{defn}{\protect\definitionname}[section]
\theoremstyle{definition}
\newtheorem{example}{\protect\examplename}[section]
\theoremstyle{remark}
\newtheorem*{rem*}{\protect\remarkname}
\theoremstyle{plain}
\newtheorem{thm}{\protect\theoremname}[section]

%%%%%%%%%%%%%%%%%%%%%%%%%%%%%% User specified LaTeX commands.
\usepackage{graphicx}
\usepackage{xcolor}
\usepackage{tikz}
\usepackage{qtree}
\usepackage[many]{tcolorbox}
\tcbuselibrary{listings}

\definecolor{mygreen}{rgb}{0,0.6,0}
\definecolor{mygray}{rgb}{0.5,0.5,0.5}
\definecolor{mymauve}{rgb}{0.58,0,0.82}

\definecolor{backcolour}{rgb}{0.95,0.95,0.92}
\definecolor{codegreen}{rgb}{0,0.6,0}

\usepackage{pgfplots}
\pgfplotsset{compat=newest}
\pgfplotsset{plot coordinates/math parser=false}
\usetikzlibrary{arrows.meta}
\usetikzlibrary{calc}
\usetikzlibrary{shapes}
\usetikzlibrary{plotmarks}
\usetikzlibrary{matrix}
\usepackage{tikz-3dplot}
\usetikzlibrary{3d}
\usepackage{fontspec}
\setmainfont[BoldFont={GARABD.ttf},ItalicFont=GARAIT.ttf]{GARA.ttf}
\hypersetup{urlcolor=blue}

\makeatother

\usepackage{polyglossia}
\setdefaultlanguage[variant=american]{english}
\providecommand{\definitionname}{Definition}
\providecommand{\examplename}{Example}
\providecommand{\remarkname}{Remark}
\providecommand{\theoremname}{Theorem}

\begin{document}

\title{Partial Differential Equations}

\author{Quasar Chunawala}
\maketitle
\begin{abstract}
The most important results and ideas in classical PDEs.
\end{abstract}

\section{Vector Calculus.}

\subsection{Vector Fields.}
\begin{defn}
A vector field on $\mathbf{R}^{n}$ is a mapping 

\begin{equation}
\mathbf{F}:X\subseteq\mathbf{R}^{n}\to\mathbf{R}^{n}
\end{equation}
\end{defn}
We usually study vector fields on $\mathbf{R}^{2}$ or $\mathbf{R}^{3}$.
In such cases, we can think of a vector field as a function that assigns
to each point $\mathbf{x}\in X$, a vector $\mathbf{F}(\mathbf{x})$
in $\mathbf{R}^{n}$, represented by a geometric arrow, whose tail
is at the point $\mathbf{x}$. This allows us to visualize vector
fields in a reasonable way. 
\begin{example}
Suppose $\mathbf{F}:\mathbf{R}^{2}\to\mathbf{R}^{2}$ is defined by
$\mathbf{F}(\mathbf{x})=\mathbf{a}$ where $\mathbf{a}$ is a constant
vector. Then, $\mathbf{F}$ assigns $\mathbf{a}$ to each point of
$\mathbf{R}^{2}$. So, we can picture $\mathbf{F}$ by drawing the
same vector (parallel, translated of course) emanating from each point
in the plane, as in the figure below. 
\end{example}
\begin{center}
\begin{tikzpicture}
\begin{axis}[
    xmin = -4, xmax = 4,
    ymin = -4, ymax = 4,
    zmin = 0, zmax = 1,
    axis equal image,
    xtick distance = 1,
    ytick distance = 1,
    view = {0}{90},
    scale = 1.25,
    title = {\bf Vector Field $F = [1,1]$},
    height=7cm,
    xlabel = {$x$},
    ylabel = {$y$},
    colormap/viridis,
]
 
\addplot3[
    quiver = {
        u = 1,
        v = 1,
        scale arrows = 0.25,
    },
    quiver/colored = {mapped color},
    -stealth,
    domain = -4:4,
    domain y = -4:4,
] {0};
 
\end{axis}
 
\end{tikzpicture}
\end{center}
\begin{example}
Let's depict $\mathbf{G}:\mathbf{R}^{2}\to\mathbf{R}^{2}$, $\mathbf{G}(x,y)=y\mathbf{i}-x\mathbf{j}$.
To understand $\mathbf{G}$ better, we need to play around a bit.
Note that:

\begin{align*}
\left\Vert \mathbf{G}(x,y)\right\Vert  & =\sqrt{y^{2}+x^{2}}=\left\Vert \mathbf{r}\right\Vert 
\end{align*}
\end{example}
where $\mathbf{r}=x\mathbf{i}+y\mathbf{j}$, the position vector of
the point $(x,y)$. From this observation, it follows that $\mathbf{G}$
has a constant length $a$ on the circle $x^{2}+y^{2}=a^{2}$. In
addition, we have:

\begin{align*}
\mathbf{r}\cdot\mathbf{G}(x,y) & =(x\mathbf{i}+y\mathbf{j})\cdot(y\mathbf{j}-x\mathbf{i})=0
\end{align*}

Hence, $\mathbf{G}(x,y)$ is always perpendicular to the position
vector of the point $(x,y)$. These facts, together with a table like
the preceding one, make it possible to see that $\mathbf{G}$ looks
like below. 

\begin{center}
\begin{tikzpicture}
\begin{axis}[
    xmin = -4, xmax = 4,
    ymin = -4, ymax = 4,
    zmin = 0, zmax = 1,
    axis equal image,
    xtick distance = 1,
    ytick distance = 1,
    view = {0}{90},
    scale = 1.25,
    title = {\bf Vector Field $F = [-y,x]$},
    height=7cm,
    xlabel = {$x$},
    ylabel = {$y$},
    colormap/viridis,
]
 
\addplot3[
	point meta = {sqrt(x^2 + y^2)},
    quiver = {
        u = {-y/sqrt(x^2 + y^2)},
        v = {x/sqrt(x^2 + y^2)},
        scale arrows = 0.25,
    },
    quiver/colored = {mapped color},
    -stealth,
    domain = -4:4,
    domain y = -4:4,
] {0};
 
\end{axis}
 
\end{tikzpicture}
\end{center}

Sometimes the scalar-valued function $f:X\subseteq\mathbf{R}^{n}\to\mathbf{R}$
is called a scalar field. One thinks of a vector field on $\mathbf{R}^{n}$
as attaching vector information (such as wind velocity) to each point
and a scalar field as attaching real number information (such as temperature
and pressure). We'll use the term scalar field only ocassionally,
but we don't want to shock you, when we do. 
\begin{example}
Let $\mathbf{r}=x\mathbf{i}+y\mathbf{j}+z\mathbf{k}$. The so-called
\textbf{inverse square }vector field in $\mathbf{R}^{3}$is a function
$\mathbf{F}:\mathbf{R}^{3}-\{0\}\to\mathbf{R}^{3}$ given by:

\begin{align*}
\mathbf{F}(x,y,z) & =\frac{c}{\left\Vert \mathbf{r}\right\Vert ^{3}}\mathbf{r}
\end{align*}

where $c$ is any non-zero constant. If the term “inverse square”
seems inappropriate for you, we'll try to convince you otherwise.
Set $\mathbf{u}=\frac{\mathbf{r}}{\left\Vert \mathbf{r}\right\Vert }$
so that $\mathbf{r}=\left\Vert \mathbf{r}\right\Vert \mathbf{u}$.
Then, $\mathbf{F}$ is given by:

\begin{align*}
\mathbf{F}(x,y,z) & =\frac{c}{\left\Vert \mathbf{r}\right\Vert ^{3}}\mathbf{r}=\frac{c}{\left\Vert \mathbf{r}\right\Vert ^{2}}\mathbf{u}
\end{align*}
Thus, $\mathbf{F}$ is a vector field whose direction at the point
$P(x,y,z)\neq(0,0,0)$ is parallel to the vector from the origin to
$P$ and whose magnitude is inversely proportional to the square of
the distance from the origin to $P$. Note, that $\mathbf{F}$ points
away from the origin if $c$ is positive and towards the origin if
$c$ is negative. 

We have seen an example of an inverse square field - namely, the Newtonian
gravitational field between two bodies. If one of the bodies is at
the origin and the other at the point $(x,y,z)$, then we have:

\begin{align*}
\mathbf{F} & =-\frac{GMm}{\left\Vert \mathbf{r}\right\Vert ^{2}}\mathbf{u}
\end{align*}

In this case, the proportionality constant $c$ is $-GMm$, which
is negative. This means that the gravitational force is attractive.
This vector field is drawn in the figure below. 
\end{example}
\begin{center}
\begin{tikzpicture}
  \begin{axis}[
    domain=-1:1,
    samples=10,
    xmin=-1,xmax=1,
    ymin=-1,ymax=1,
    zmin=-1,zmax=1,
    ]
    \pgfplotsinvokeforeach{-1,-.5,0,.5,1}{
      \addplot3[cyan,quiver,-stealth,
      point meta={sqrt((x)^2+(y)^2+(z)^2)},
      quiver={
        u={x/sqrt((x)^2+(y)^2+(z)^2)},
        v={y/sqrt((x)^2+(y)^2+(z)^2)},
        w={z/sqrt((x)^2+(y)^2+(z)^2)},
        colored,scale arrows=.1}]
      (x,y,#1);
    }
  \end{axis}
\end{tikzpicture}
\end{center}

\subsection{Gradient fields and Potentials.}

Inverse square fields are interesting not only for their origin in
physics, but also because they are examples of gradient fields. 
\begin{defn}
(\textbf{Gradient Field}) A gradient field on $\mathbf{R}^{n}$ is
a vector field $\mathbf{F}:X\subseteq\mathbf{R}^{n}\to\mathbf{R}^{n}$
such that $\mathbf{F}$ is the gradient of some differentiable scalar-valued
function $f:X\to\mathbf{R}$. That is:

\begin{align*}
\mathbf{F}(\mathbf{x}) & =\nabla f(\mathbf{x})
\end{align*}

at all $\mathbf{x}$ in $X$. The function $f$ is called a (scalar)
\textbf{potential function }for the vector field $\mathbf{F}$. To
see what this means in the case of the inverse square field, we write
out the components of $\mathbf{F}$ explicitly:

\begin{align*}
\mathbf{F} & =\frac{c}{\left\Vert \mathbf{r}\right\Vert ^{2}}\mathbf{u}=\frac{c}{(x^{2}+y^{2}+z^{2})}\cdot\frac{x\mathbf{i}+y\mathbf{j}+z\mathbf{k}}{\sqrt{x^{2}+y^{2}+z^{2}}}
\end{align*}

since the position vector $\mathbf{r}=x\mathbf{i}+y\mathbf{j}+z\mathbf{k}$
and $\mathbf{u}=\mathbf{r}/\left\Vert \mathbf{r}\right\Vert $.

Let 

\begin{align*}
f(x,y,z) & =-\frac{c}{\sqrt{x^{2}+y^{2}+z^{2}}}=-\frac{c}{\left\Vert \mathbf{r}\right\Vert }
\end{align*}

The gradient vector for the function $f$ is:

\begin{align*}
\nabla f(x,y,z) & =\left[\begin{array}{ccc}
\frac{\partial f}{\partial x} & \frac{\partial f}{\partial y} & \frac{\partial f}{\partial z}\end{array}\right]\\
 & =-c\left[\begin{array}{ccc}
-\frac{1}{2}\cdot\frac{2x}{(x^{2}+y^{2}+z^{2})^{3/2}} & -\frac{1}{2}\cdot\frac{2y}{(x^{2}+y^{2}+z^{2})^{3/2}} & -\frac{1}{2}\cdot\frac{2z}{(x^{2}+y^{2}+z^{2})^{3/2}}\end{array}\right]\\
 & =\frac{c}{(x^{2}+y^{2}+z^{2})^{3/2}}(x\mathbf{i}+y\mathbf{j}+z\mathbf{k})\\
 & =\mathbf{F}(\mathbf{x})
\end{align*}

Hence, $f(x,y,z)$ is the potential function for the field $\mathbf{F}$.
\end{defn}
\begin{rem*}
In Physics and Engineering, a negative sign is often introduced in
the definition of a potential function (so that a potential function
$g$ for a vector field $\mathbf{F}$ is one such that $\mathbf{F}=-\nabla g$).
The motivation behind such a convention is that in physical applications,
it is desirable to have the potential function represent potential
energy in some sense. For example, in the case of the gravitational
field $\mathbf{F}=-(GMm/\left\Vert \mathbf{r}\right\Vert ^{2})\mathbf{u}$,
a physicist would take the potential function to be $U(\mathbf{x})=-GMm/\left\Vert \mathbf{r}\right\Vert $.
The advantage to the physicist in doing so is that the physicist's
potential function increases with increasing $\left\Vert \mathbf{r}\right\Vert $.
This corresponds to the notion that the greater the distance between
the two bodies, the greater should be stored gravitational potential
energy.
\end{rem*}

\subsubsection{Gradient vector and the Level Sets.}
\begin{thm}
Let $X\subseteq\mathbf{R}^{n}$ be an open set and $f:X\to\mathbf{R}$
be a function(surface) of class $C^{1}$. If $\mathbf{x}_{0}$ is
any point on the level set $S=\{\mathbf{x}|f(\mathbf{x})=c\}$, then
the gradient vector $\nabla f(\mathbf{x})$ is perpendicular to the
level set $S$ at the point $\mathbf{x}_{0}$.
\end{thm}
%
\begin{proof}
We need to establish the following : if $\mathbf{v}$is any vector
tangent to $S$ at the point $\mathbf{x}_{0}$, then $\nabla f(\mathbf{x}_{0})$
is orthogonal to $\mathbf{v}$. 

By a tangent vector to $S$ at $\mathbf{x}_{0}$, we mean that $\mathbf{v}$
is the velocity vector of a curve $C$ that lies in $S$ and passes
through $\mathbf{x}_{0}$. Imagine that $f:X\subseteq\mathbf{R}^{3}\to\mathbf{R}$
is defined as $w=f(x,y,z)$. Thus, the graph(plot) of $f$ lives in
$\mathbf{R}^{4}$. The set $S=\{\mathbf{x}|f(\mathbf{x})=c\}$ is
a level surface of $f$ that lives in $\mathbf{R}^{3}$. The situation
is pictured below:

\begin{center}
\begin{tikzpicture}
	\filldraw [thick,fill=yellow!20!white] (0,0) .. controls (0,2.0) .. (1,4.0) .. controls (3.5,4.2) and (6,4) .. (8,3.0) .. controls (7,0) .. (7,-1) .. controls (6,-0.2) and (3,1) .. (0,0);
	\filldraw (1.0,1.0) circle(0.03cm);
	\draw (1.0,1.0) .. controls (2.0,1.5) and (3.0,2.3) .. (4,2.0) .. controls (5.0,1.7) and (6.0,1.2) .. (7.0,3.0);
	\filldraw (3.9,2.05) circle(0.03cm) node [below] {$\mathbf{x}_0$};
	\draw [->,blue,thick] (3.9,2.05) -- (5.0,2.05) node [below] {$\mathbf{v}$};
	\draw [->,blue,thick] (3.9,2.05) -- (3.9,3) node [right] {$\nabla f(\mathbf{x}_0)$};
	\draw (3.9,2.2) -- (4.1,2.2) -- (4.1,2.05);
	\filldraw (7.0,3.0) circle(0.03cm) node [below] {$C$};
\end{tikzpicture}
\end{center}

Thus, let the curve $C$ be given parametrically by $\mathbf{x}(t)=(x_{1}(t),x_{2}(t),\ldots,x_{n}(t))$,
where $a<t<b$ and $\mathbf{x}(t_{0})=\mathbf{x}_{0}$. Since the
curve $C$ is contained in $S$, we have: 

\begin{align*}
f(\mathbf{x}(t)) & =f(x_{1}(t),x_{2}(t),\ldots,x_{n}(t))=c
\end{align*}

Hence, 

\begin{equation}
\frac{d}{dt}[f(\mathbf{x}(t))]=\frac{d}{dt}[c]=0
\end{equation}

On the other hand, the chain rule applied to the composite function
$f(\mathbf{x}(t)):[a,b]\to\mathbf{R}$ tells us:

\begin{equation}
\frac{d}{dt}[f(\mathbf{x}(t))]=\nabla f(\mathbf{x})\cdot\mathbf{x}'(t)
\end{equation}

At $t=t_{0}$, we have:

\begin{align*}
\nabla f(\mathbf{x}_{0})\cdot\mathbf{x}'(t_{0}) & =\nabla f(\mathbf{x}_{0})\cdot\mathbf{v}=0
\end{align*}

But, the velocity vector $\mathbf{v}$ of the curve $C$ at $\mathbf{x}_{0}$
is always parallel to $C$. This closes the proof.
\end{proof}
It follows that, if $\mathbf{F}$ is a gradient vector field on $\mathbf{R}^{n}$,
$\mathbf{F}(\mathbf{x})$ must be perpendicular to the level sets
of the potential function of $\mathbf{F}$ containing the point $\mathbf{x}$.
If $f$ is such a potential function, the level set $\{\mathbf{x}:f(\mathbf{x})=c\}$
is called an \textbf{equipotential set }(or an equipotential surface)
of the vector field $\mathbf{F}$. Everytime we look at weather-maps,
the curves of constant barometric pressure (called isobars) or constant
temperature (isotherms) are drawn. Perpendicular to the equipotential
lines are associated gradient vector fields that point in the direction
of the most rapid increase of pressure or temperature.

The gravitational pull of the earth is a force field. Its equipotential
surface is a sphere. 

\subsubsection{Flow-lines of vector fields.}

When we draw a sketch of a vector field in $\mathbf{R}^{2}$ or $\mathbf{R}^{3}$,
it is easy to imagine that the arrows represent the velocity of some
fluid moving in space. It's natural to let the arrows blend into complete
curves. What you're doing analytically, is drawing paths whose velocity
vectors coincide with those of the vector field. 
\begin{defn}
A \textbf{flow-line} of a vector field $\mathbf{F}:X\subseteq\mathbf{R}^{n}\to\mathbf{R}^{n}$
is a differentiable path $\mathbf{x}:I\to\mathbf{R}^{n}$ such that:

\begin{align*}
\mathbf{x}'(t) & =\mathbf{F}(\mathbf{x}(t))
\end{align*}

That is, the velocity vector of $\mathbf{x}$ at time $t$ is given
by the value of the vector field $\mathbf{F}$ at the point on $\mathbf{x}$
at time $t$.
\end{defn}
\begin{example}
We calculate the flow lines of the constant vector field $\mathbf{F}(x,y,z)=2\mathbf{i}-3\mathbf{j}+\mathbf{k}$.

A picture of this vector field below makes it easy to believe that
the flow lines are indeed straight line paths. Indeed, if $\mathbf{x}(t)=(x(t),y(t),z(t))$
is a flow-line, then by definition, we must have:

\begin{align*}
\mathbf{x}'(t)=(x'(t),y'(t),z'(t)) & =(2,-3,1)=\mathbf{F}(\mathbf{x}(t))
\end{align*}

\begin{center}
\begin{tikzpicture}
  \begin{axis}[
    domain=-1:1,
    samples=10,
    xmin=-1,xmax=1,
    ymin=-1,ymax=1,
    zmin=-1,zmax=1,
    ]
    \pgfplotsinvokeforeach{-1,-.5,0,.5,1}{
      \addplot3[cyan,quiver,-stealth,
      point meta={sqrt((2)^2+(3)^2+(1)^2)},
      quiver={
        u={2},
        v={-3},
        w={1},
        colored,scale arrows=.1}]
      (x,y,#1);
    }
  \end{axis}
\end{tikzpicture}
\end{center}

Equating the components, we see that:

\begin{align*}
x'(t) & =2\\
y'(t) & =-3\\
z'(t) & =1
\end{align*}

These differential equations are readily solved by integration; we
obtain:

\begin{align*}
x(t) & =2t+x_{0}\\
y(t) & =-3t+y_{0}\\
z(t) & =t+z_{0}
\end{align*}

where $x_{0},y_{0},z_{0}$ are arbitrary constants. Hence, as expected,
we obtain the parametric equations for a straight line path passing
through an arbitrary point $(x_{0},y_{0},z_{0})$ with the velocity
vector $(2,-3,1)$.
\end{example}
%

\subsection{Gradient, Divergence, Curl and the Del Operator.}

\subsubsection{The Del Operator.}

The del operator, denoted $\nabla$ is an odd creature. It leads a
double life as both a differential operator and a vector. In Cartesian
coordinates on $\mathbf{R}^{3}$, del is defined by the curious expression:

\begin{equation}
\nabla=\mathbf{i}\frac{\partial}{\partial x}+\mathbf{j}\frac{\partial}{\partial y}+\mathbf{k}\frac{\partial}{\partial z}
\end{equation}

The empty partial derivatives are the components of a vector that
awaits suitable scalar and vector fields on which to act. 

For example, if $f:X\subseteq\mathbf{R}^{3}\to\mathbf{R}$ is a differentiable
function (scalar field), the gradient of $f$ may be considered to
be the result of multiplying the vector $\nabla$ by the scalar $f$,
except that when we \emph{multiply} each component of $\nabla$ by
$f$, we actually compute the appropriate partial derivative:

\begin{align*}
\nabla f(x,y,z) & =\left(\mathbf{i}\frac{\partial}{\partial x}+\mathbf{j}\frac{\partial}{\partial y}+\mathbf{k}\frac{\partial}{\partial z}\right)f(x,y,z)=\frac{\partial f}{\partial x}\mathbf{i}+\frac{\partial f}{\partial y}\mathbf{j}+\frac{\partial f}{\partial z}\mathbf{k}
\end{align*}

The del operator can also be defined in $\mathbf{R}^{n}$ for arbitrary
$n$. If we take $x_{1},x_{2},\ldots,x_{n}$ to be the coordinates
for $\mathbf{R}^{n}$, then del is simply:

\begin{align*}
\nabla & =\frac{\partial}{\partial x_{1}}\mathbf{e}_{1}+\frac{\partial}{\partial x_{2}}\mathbf{e}_{2}+\ldots+\frac{\partial}{\partial x_{n}}\mathbf{e}_{n}
\end{align*}


\subsubsection{The divergence of a vector field.}

Whereas taking the gradient of a scalar field yields a vector field,
the process of taking the divergence does just the opposite: it turns
a vector field into a scalar field. 
\begin{defn}
\label{def:divergence-of-a-vector-field}Let $\mathbf{F}:X\subseteq\mathbf{R}^{n}\to\mathbf{R}^{n}$
be a differentiable vector field. Then, the \textbf{divergence} of
$\mathbf{F}$, denoted $\text{div}\mathbf{F}$or $\nabla\cdot\mathbf{F}$
(del dot $\mathbf{F}$) is the scalar field:

\begin{equation}
\text{div}\mathbf{F}=\nabla\cdot\mathbf{F}=\frac{\partial F_{1}}{\partial x_{1}}+\frac{\partial F_{2}}{\partial x_{2}}+\ldots+\frac{\partial F_{n}}{\partial x_{n}}
\end{equation}

where $x_{1},x_{2},\ldots,x_{n}$ are Cartesian coordinates for $\mathbf{R}^{n}$
and $F_{1},F_{2},\ldots,F_{n}$ are the component functions of $\mathbf{F}$.

It is essential that Cartesian coordinates be used in the formula
of the definition (\ref{def:divergence-of-a-vector-field}). 
\end{defn}
\begin{example}
If $\mathbf{F}=x^{2}y\mathbf{i}+xz\mathbf{j}+xyz\mathbf{k}$, then:

\begin{align*}
\text{div}\mathbf{F} & =\frac{\partial}{\partial x}(x^{2}y)+\frac{\partial}{\partial y}(xz)+\frac{\partial}{\partial z}(xyz)\\
 & =2xy+xy\\
 & =3xy
\end{align*}

The notation for the divergence involving the dot product and the
del operator is especially apt: if we write:

\begin{align*}
\mathbf{F} & =F_{1}\mathbf{e}_{1}+F_{2}\mathbf{e}_{2}+\ldots+F_{n}\mathbf{e}_{n}
\end{align*}

then,

\begin{align*}
\nabla\cdot\mathbf{F} & =\left(\mathbf{e}_{1}\frac{\partial}{\partial x_{1}}+\ldots+\mathbf{e}_{n}\frac{\partial}{\partial x_{n}}\right)\cdot(F_{1},\ldots,F_{n})\\
 & =\frac{\partial F_{1}}{\partial x_{1}}+\ldots+\frac{\partial F_{n}}{\partial x_{n}}
\end{align*}

Intuitively, the value of the divergence of a vector field at a particular
point gives a measure of the \emph{net mass flow} or \emph{flux density}
of the vector field in to or out of that point. To understand what
such a statement means, imagine that the vector field $\mathbf{F}$
represents the velocity of a fluid. If $\nabla\cdot\mathbf{F}$ is
zero at a point, then the rate at which fluid is flowing into that
point is equal to the rate at which fluid is flowing out. Positive
divergence at a point signifies more fluid flowing out than in, while
negative divergence signifies just the opposite. We will make these
assertions more precise, even prove them, when we will have some integral
vector calculus at our disposal. For now, we can remark, that a vector
field $\mathbf{F}$ such that $\text{div \ensuremath{\mathbf{F}}=}\nabla\cdot\mathbf{F}=0$
everywhere is called \textbf{incompressible }or \textbf{solenoidal}. 
\end{example}
%
\begin{example}
The vector field $\mathbf{F}=x\mathbf{i}+y\mathbf{j}$ has:

\begin{align*}
\text{div}\mathbf{F} & =\nabla\cdot\mathbf{F}\\
 & =\frac{\partial}{\partial x}(x)+\frac{\partial}{\partial y}(y)\\
 & =2
\end{align*}
\end{example}
\begin{center}
\begin{tikzpicture}
\begin{axis}[
    xmin = -4, xmax = 4,
    ymin = -4, ymax = 4,
    zmin = 0, zmax = 1,
    axis equal image,
    xtick distance = 1,
    ytick distance = 1,
    view = {0}{90},
    scale = 1.25,
    title = {\bf Vector Field $\mathbf{F} = [x,y]$},
    height=7cm,
    xlabel = {$x$},
    ylabel = {$y$},
    colormap/viridis,
	samples = 8,
]
 
\addplot3[
	point meta = {sqrt(x^2 + y^2)},
    quiver = {
        u = {x},
        v = {y},
        scale arrows = 0.25,
    },
    quiver/colored = {mapped color},
    -stealth,
    domain = -4:4,
    domain y = -4:4,
] {0};
 
\end{axis}
 
\end{tikzpicture}
\end{center}

At any point in $\mathbf{R}^{2}$, the arrow whose tail is at that
point is longer than the arrow who head is there (indicated by the
color in the above plot). Hence, there is a greater flow \emph{away}
from each point, than into it; that is \emph{diverging} at each point.

The vector field $\mathbf{G}=-x\mathbf{i}-y\mathbf{j}$ points in
the direction opposite to the vector field $\mathbf{F}$, and it should
be clear how $\mathbf{G}$'s divergence of $-2$ is reflected in the
below plot.

\begin{center}
\begin{tikzpicture}
\begin{axis}[
    xmin = -4, xmax = 4,
    ymin = -4, ymax = 4,
    zmin = 0, zmax = 1,
    axis equal image,
    xtick distance = 1,
    ytick distance = 1,
    view = {0}{90},
    scale = 1.25,
    title = {\bf Vector Field $\mathbf{G} = [-x,-y]$},
    height=7cm,
    xlabel = {$x$},
    ylabel = {$y$},
    colormap/viridis,
	samples = 8,
]
 
\addplot3[
	point meta = {sqrt(x^2 + y^2)},
    quiver = {
        u = {-x},
        v = {-y},
        scale arrows = 0.25,
    },
    quiver/colored = {mapped color},
    -stealth,
    domain = -4:4,
    domain y = -4:4,
] {0};
 
\end{axis}
 
\end{tikzpicture}
\end{center}
\begin{example}
The constant vector field $\mathbf{F}(x,y,z)=\mathbf{a}$ is incompressible,
since if $\mathbf{a}=(a_{x},a_{y},a_{z})$, we have:

\begin{align*}
\nabla\cdot\mathbf{F} & =\frac{\partial}{\partial x}(a_{x})+\frac{\partial}{\partial x}(a_{y})+\frac{\partial}{\partial z}(a_{z})\\
 & =0
\end{align*}

\end{example}

\end{document}
